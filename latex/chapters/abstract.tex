% !TeX root = ../main.tex

\ustcsetup{
    keywords = {
        多模态学习, 层级多标签分类
    },
    keywords* = {
        Multimodal Learning, Hierarchical Multilabel Classification
    },
}

\begin{abstract}
    随着线上教育模式的兴起,对教育类视频进行知识点预测逐渐成为一个基础而又重要的任务。
    % 然而现有的教育视频知识点预测方法通常忽略了知识点的层级结构,使得预测结果往往不满足知识点的层级一致性。
    % 同时,现有的方法也不具有良好的结果可解释性,通常只给出了教育视频所涉及到的知识点,而对于这些知识点出现在视频中的哪些片段并没有给出解释。
    % 通常教育类视频至少包含视频与字幕文本两种模态的信息,同时视频信息中
    % 教育类视频通常具有时长较长、涉及知识点较多的特点,简单地将整个视频的全量信息输入到深度学习模型中进行预测会造成计算负担过重的问题,
    教育类视频中包含图像、文本等多种模态的信息,且知识点体系通常可以组织成一个有向无环图或树状的层级结构,因此对教育视频进行知识点预测本质上是一个多模态的层级多标签分类任务。
    教育类视频通常具有时长较长、涉及知识点较多的特点,利用好视频中的大量信息进行较为完整的知识点预测,以及将每个预测出的知识点在原视频中进行定位都是具有挑战性的问题。
    同时由于教育视频知识点预测的任务本质,如何有效地将视频中的多种模态信息进行结合,如何能够在保证分类性能的同时又尽量满足层级分类的要求,也都是本文要面临的挑战。
    为了解决这些问题,本文提出了一种基于视频内容的关键帧抽取算法,以及适用于层级标签预测的深度学习网络模型,该模型能够结合视频、文本特征,逐层地进行知识点预测。
    具体而言,首先本文通过关键帧抽取算法去除视频中的冗余信息,得到一系列视频关键帧,同时根据关键帧之间的差异性对视频与字幕进行分块。
    然后,本文通过多模态特征提取层从关键帧和字幕文本中提取特征向量,再分块送入基于注意力机制的层级网络中,
    提取出 1) 知识点层级结构与教育视频之间的关联以及 2) 教育视频内部多模态信息之间的关联,最后得到视频块的局部表征和知识点的局部预测结果。
    最后通过混合预测模块融合局部表征和局部预测结果,得到每一个视频块以及整个教育视频的知识点预测结果。
\end{abstract}

\begin{abstract*}
    With the rise of the online education, knowledge prediction for educational videos has gradually become a basic and important task.
    % Educational videos usually have the characteristics of long duration and high information density,
    % so simply putting the full information of the entire video into the deep learning model will cause the problem of excessive computational burden,
    % % {\color{red} and the results of knowledge prediction are interpretable.}
    % and the low interpretability of the knowledge prediction results,
    % that is, no explanation is given for where these knowledge appear in the video.
    Educational videos contain multimodal information such as images and texts, and the knowledge system can usually be organized into a Directed Acyclic Graph(DAG) or a Tree-like structure.
    Therefore, knowledge prediction for educational videos is essentially a multimodal Hierarchical Multi-label Classification task.
    Educational videos usually have the characteristics of being long and involving much knowledge,
    so it is challenging to make good use of a large amount of information in the video to predict knowledge completely, and to locate each predicted knowledge in the original video.
    At the same time, due to the essence of the task of educational video knowledge prediction,
    % {\color{red} how to effectively combine various modal information in the video,
    % and how to ensure the classification performance while meeting the requirements of hierarchical classification as much as possible are all challenges we have to face.}
    it is also challenging to effectively combine multimodal information in the video, and get good classification performance while meeting the requirements of hierarchical consistency.
    In order to solve these problems, in this paper, we propose a Content-Based keyframe extraction algorithm and a deep learning model suitable for hierarchical knowledge prediction,
    which can combine video and text features to predict knowledge layer by layer.
    Specifically, we first remove redundant information in the video through the keyframe extraction algorithm,
    obtain a series of video keyframes, and divide the video and subtitles into sections according to the differences between the keyframes.
    Then, we extract features from keyframes and subtitles through a multimodal representation layer, and then put them into an attention-based hierarchical network to extract
    1) the association between the knowledge hierarchy and educational videos and
    2) correlation between multimodal information within educational videos,
    and finally obtaining local representations and local predictions of video sections.
    Finally, the Hybrid Predicting Layer fuses the local representations and local predictions to compute the final knowledge predictions of each video section and the entire educational video.
    % {\color{red} and the knowledge predictions of each video section and the entire educational video are obtained.}
\end{abstract*}
