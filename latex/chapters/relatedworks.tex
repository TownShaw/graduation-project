\chapter{相关工作}

\section{多模态机器学习}
    多模态机器学习主要是用于处理输入数据形式为多种模态的问题,比如输入数据形式为图像和文本\cite{Frome2013DeViSEAD}、视频和文本\cite{Xu2015JointlyMD, Bojanowski2015WeaklySupervisedAO};
    或是需要在保持输入数据所包含信息不变的情况下,生成另一种模态的等价数据信息,比如视频字幕生成\cite{Austin2017TranslatingVT}、图片描述文本生成\cite{Mao2015DeepCW}等。
    多模态机器学习所面临的挑战主要可分为以下五类\cite{Baltruaitis2019MultimodalML}:多模态特征表示、多模态特征翻译、多模态特征对齐、多模态特征融合以及多模态共同学习。
    在对教育类视频进行知识点预测时,我们主要通过多模态特征表示和多模态特征融合来处理输入的多模态视频数据。
    多模态特征表示主要可以分为两种方式:联合特征表示和协同特征表示。
    联合特征表示会将各个模态的特征向量映射到同一个特征表示空间,并将映射得到的特征向量作为多模态数据的最终特征表示。
    获取各个模态的特征向量以及对特征向量进行投影可以通过神经网络来实现,
    Y. Mroueh等人\cite{Mroueh2015DeepML}通过两个深度网络分别提取出音频与视频特征,然后直接将这两部分特征向量进行拼接,从而完成多模态特征表示。
    S. Antol等人\cite{Agrawal2015VQAVQ}与W. Ouyang等人\cite{Ouyang2014MultisourceDL}则是在使用深度网络分别提取出各个模态的特征向量之后,使用多层感知机将这些特征投影到同一个特征空间中。
    与联合特征表示不同的是,协同特征表示会在不同的特征空间中分别处理各模态的特征向量,但会在不同模态的特征向量之间加上相似性约束,从而构成了特征协同空间。
    比如A. Frome等人\cite{Frome2013DeViSEAD}使用距离函数来衡量图像特征和文本特征之间的相似性,图像与文本之间越相似,则距离函数值越小。
    在完成多模态特征表示之后,为了进一步提取出各个模态特征之间的关联信息,需要再对提取出的特征信息进行多模态特征融合。
    多模态特征融合可以大致分为三种融合模式:Early 模式、Late 模式与 Hybrid 模式。
    Early 模式的多模态特征融合会在使用特征信息作出某一个决定之前将各个模态信息进行融合,而 Late 模式的融合方式则相反,会先使用各个模态信息分别作出某个决定,然后再将结果进行综合。
    而 Hybrid 模式则是综合 Early 和 Late 两种模式,来对多模态的特征信息进行融合。
    H. Gao等人\cite{Gao2015AreYT}在使用联合特征表示将图像和文本信息投影到同一个特征空间之后,采取直接将这两个特征向量进行相加的方式完成多模态特征融合。
    M. Malinowski等人\cite{Malinowski2015AskYN}则是在 LSTM 中完成对图像和文本特征信息的融合,将多模态信息融入到 LSTM 的隐向量中向后传播。
    % {\color{red} 但是这种多模态特征融合方式并不能较好地捕获到图像与某一段局部文本信息之间的关系。}


\section{层级多标签分类}
    层级多标签分类领域的分类方法大致可以分为两类:不考虑任何层级标签体系本身进行平坦化分类的方式,以及结合层级标签体系的信息进行分类的方式。
    平坦化分类方式通过忽略层级结构的方式,将层级多标签分类问题转化为普通的多标签分类问题。
    平坦化分类方式通常只考虑层级体系中最后一层的叶子节点标签,在对这些标签分类完成之后,自动地将叶子标签中标记为正的标签所有的祖先标签都标记为正,从而得到一个满足 TPR 规则的预测结果。
    尽管平坦化分类方式保证满足 TPR 规则,不会引起层级不一致,但是由于完全忽略层级标签体系内含的信息,从而通常会导致较低的分类效果。
    同时,由于平坦化分类方式仅仅考虑最后一层的叶子标签节点,在面对不强制分类到叶子标签的问题时,平坦化分类将变得不再适用\cite{Silla2010ASO}。
    结合了层级标签体系信息进行分类的方式也可以再细分为两类,一类是将层级多标签分类问题分解为多个更小的普通多标签分类问题,因为每次分类都只考虑到了局部的层级结构信息,因此称为局部分类方式;
    另一类则是一次性考虑到全部的层级标签体系结构,仅用一次分类就完成整个层级多标签分类问题。
    在局部分类方式中,Cesa-Bianchi等人\cite{CesaBianchi2004IncrementalAF}在每一个标签节点上都训练了一个 SVM 二分类器,将层级多标签分类问题转化为了若干个二分类问题。
    同时为了满足 TPR 规则,一个标签节点上的 SVM 分类器当且仅当其父标签被标记为正时才会进行迭代更新,这样就组成了一个层级 SVM 分类器。
    Secker等人\cite{Secker2007AnEC}则是使用了一种自上而下的层级分类方式,与 Cesa-Bianchi等人不同的是,Secker等人仅在具有子标签的标签节点上训练分类器,将层级多标签分类问题转化为了若干个普通多标签分类问题。
    同时,Secker等人还尝试在不同的标签节点上使用不同的分类器进行多标签分类,依此期望获得更好的分类效果。
    Cerri等人\cite{Cerri2016ReductionSF}则提出了一种基于神经网络的 HMC-LMLP 模型,通过每层使用一个多层感知机来对该层所有标签进行分类。
    在全局分类方式中,Vens等人\cite{Vens2008DecisionTF}通过使用一个全局的决策树 Clus-HMC 来对整个树型的层级标签体系分类。
    近年,Wehrmann等人\cite{Wehrmann2018HierarchicalMC}与 Huang等人\cite{Huang2019HierarchicalMT}则结合了局部分类方式和全局分类方式,
    分别提出了基于神经网络的深度模型 HMCN 和 HARNN,使用混合分类预测的方式来解决层级多标签分类问题。
    同时 HARNN 还注意到了文本数据与标签信息之间可能存在的关联,将层级标签体系也视作一种模态的信息,并使用注意力机制捕获文本与标签之间的关联信息,从而获得了更好的分类效果。
